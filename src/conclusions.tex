\section{Future Work and Decisions}\label{sec:conclusions}

The development of the \texttt{v2} prototype of the columnar CP tools is ongoing and will continue to be publicly developed and benchmarked with future updates to the community from ATLAS.
The \texttt{v1} and \texttt{v2} prototypes have already demonstrated that adopting a columnar implementation for the backend CP tools allows for a columnar analysis paradigm to be possible, and that the ongoing integration of \texttt{nanobind} bindings bridges the gap between \texttt{C++} and Python for high performance analysis while allowing freedom of Pythonic API design to allow for higher level analysis thinking by end users.
This development is critical for a full columnar analysis ATLAS demonstrator that is able to interface fully with the scientific Python and PyHEP ecosystems.

When the ATLAS CP tools were created 10-15 years ago during Run 1, they were designed to be as framework independent as possible, be able to be run by themselves, and to interface with the ATLAS EDM.
In the time since, they have been battle tested, and through through hundreds of analyses are now extremely well understood and provide excellent physics performance.
This creates a strong desire in ATLAS to maintain these valuable tools and their existing functionality, making the rewrite cost of them to a \texttt{correctionlib}~\cite{correctionlib_2024} paradigm currently too high.
The addition of columnar support to the CP tools for the new analysis model though requires cracking open the ``black box'' implementations of the tools.
Being confronted with legacy code decisions further highlights the columnar prototype design decisions and the design opportunities during the tool migrations.
It also prompts more ambitious questions, like what would be required to make the ATLAS CP tools a standalone \texttt{pip}~\cite{pip_github} or \texttt{conda}~\cite{conda_software} installable Python package?
While this might seem like an ambition for Run 5 of the LHC or beyond, given the significant improvements in packaging technology in the Python ecosystem, like \texttt{scikit-build-core}~\cite{Schreiner_Scikit-build-core_2024}, this proposal is perhaps achievable on a significantly shorter time scale.
As an example, using this strategy, the ROOT team is currently researching the ability to distribute ROOT as a Python package~\cite{Padulano:CHEP_2024}.
If these approaches are successful, it will allow for a greater level of modularity in the way ATLAS analyses are performed and a path towards greater interoperability with other tools as the ATALS CP tools will have become another viable tool in the broader PyHEP ecosystem.
