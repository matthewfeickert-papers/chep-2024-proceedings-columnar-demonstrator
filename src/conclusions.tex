\section{Preliminary Performance Tests and Future Decisions and Work}\label{sec:conclusions}

During the ongoing refactor for the \texttt{v2} prototype, preliminary integrated benchmarks have been created to measure the time spent in each tool per event (excluding I/O) in comparison with the xAOD model.
While direct one-to-one comparisons are not possible given the inherent differences in data processing, the tests have been designed to be as close as possible.
The benchmarks compare the same version of each cP tool, only use the \texttt{C++} code (no Python is involved so as to isolate the \texttt{C++} performance), and the time for the xAOD model includes the event store access overhead (which is per-event for the xAOD model and per-batch for the columnar model).
The time for I/O and connecting columns is also not included in the performance comparisons, as this has not been optimized in the current tests and will not provide useful information, and so are removed from the benchmark.
The benchmarks show substantial speedups for the migrated tools with the columnar implementations ranging from being \emph{2-4x faster} than the xAOD interface.
The specific reasons for the speedups are currently being investigated fully, but preliminary checks show a relation with EDM access (columnar tools need to access the EDM once per event batch).

ATLAS CP tools were created 10-15 years ago to run in an analysis framework.
Battle tested, extremely well understood, excellent physics performance, strong desire to be maintained.
Rewrite cost is currently too high across collaboration to move to \texttt{correctionlib} paradigm.
Columnar cracks open ``black box'' implementations of tools for the new analysis model.
Legacy code decisions highlight columnar prototype design decisions and opportunities during tool migration.
Raises the question: ``What would it take to get to \texttt{python -m pip install atlascp}?''
Columnar prototype explores these possibilities.
Steps beyond: Modularization to level that allows packaging with \texttt{scikit-build-core}.
